\documentclass{beamer}
\usetheme{default}

\usepackage{calc}
\usepackage{forloop}

\usepackage{tikz}

\usepackage{lstcustom}

% Important modifications of the custom file.  Escaping the symbol
% dollar sign ($) in the code is necessary for compilation.

\lstdefinestyle{eclipseNoBox}{
  basicstyle={\lstfontfamily},
  emphstyle=\bfseries,
  keywordstyle=\color{keywordColor}\bfseries,
  commentstyle=\markupComments,
  stringstyle=\color{stringColor},
  numberstyle=\color{lineNumberColor}\lstfontfamily,
  morecomment=[s][\markupJavadocs]{/**}{*/}, % For Javadoc comments
  showstringspaces=false,
  numbers=none,
}

\lstset{
  language=C++,
  style=eclipseNoBox,
  showspaces=false, 
  numbers=left,
  frame=lrtb,
  captionpos=b,
  breaklines=true,
  % From http://tex.stackexchange.com/questions/116534/lstlisting-line-wrapping
  postbreak=\raisebox{0ex}[0ex][0ex]{\ensuremath{\color{red}\hookrightarrow\space}},
  escapechar=\$
}


\title{C++ Templates}
\subtitle{Mark Royer}
\date{}



% Style for answers using exampleblock
\setbeamertemplate{blocks}[rounded][shadow=false]
\addtobeamertemplate{block begin}{\pgfsetfillopacity{0.8}}{\pgfsetfillopacity{1}}
\setbeamercolor*{block title example}{fg=blue!50,bg= blue!10}
\setbeamercolor*{block body example}{bg= blue!5}


\usetikzlibrary{fit,calc,shadows}

% Define styles for balloons and lines
\tikzstyle{line} =    [draw]%, width=3pt]%rounded corners=3pt, -latex]
\tikzstyle{balloon} = [draw, fill=blue!20, opacity=0.4, inner sep=5pt, rounded corners=2pt]
\tikzstyle{comment} = [draw, fill=blue!70, text=white, text width=3cm, minimum height=1cm, rounded corners, drop shadow, align=left]

% Command to place a TikZ anchor at the current position
\newcommand{\tikzmark}[1]{%
  \tikz[overlay,remember picture,baseline] \coordinate (#1) at (0,0) {};}

% Command to draw a balloon over two anchors
\newcommand<>{\balloon}[3][balloon]{%
  \coordinate (c) at ($(#2)+(0, 5.445pt)$);
  \node#4 (#1) [balloon, fit=(#3) (c)] {};}

% Start for listing highlighting
\newcommand\eh{%
\tikz[remember picture,overlay]{
  \coordinate (end highlight){};%
  \draw[yellow,line width=14pt,opacity=0.3]
    ([yshift=4pt]begin highlight) -- ([yshift=4pt]end highlight);}%
}
\newcommand{\bh}{%
  \tikz[remember picture]\coordinate(begin highlight){};}

\newcommand<>{\highlight}[2]{
  \draw#3[yellow,line width=14pt,opacity=0.3]
    ([yshift=4pt]#1) -- ([yshift=4pt]#2);%
}

% End for listing highlighting

% According to this
% http://tex.stackexchange.com/questions/165929/semiverbatim-with-tikz-in-beamer
% The following may cause issues in some documents, but it won't
% compile without it!
\makeatletter
\global\let\tikz@ensure@dollar@catcode=\relax
\makeatother

\begin{document}

%%%%%%%%%%%%%%%%%%%%%%%%%%%%%%%%%%%%%%%%%%%%%%%%%%%%%%%%%%%%

\begin{frame}
  \titlepage
\end{frame}

%%%%%%%%%%%%%%%%%%%%%%%%%%%%%%%%%%%%%%%%%%%%%%%%%%%%%%%%%%%%

\begin{frame}
  %\fontsize{16}{26}\selectfont
  Features in C++ that help to generalize code
  \begin{itemize}
    \item \textbf{Overload functions}
    \item \textbf{Template functions}
    \item \textbf{Template classes}
  \end{itemize}
\end{frame}

%%%%%%%%%%%%%%%%%%%%%%%%%%%%%%%%%%%%%%%%%%%%%%%%%%%%%%%%%%%%

\begin{frame}[fragile]{Overloaded functions}
C++ allows you to \textbf{overload} functions using the same name with
different parameter lists.

For example, you can write two swap functions in your program: one to
swap \tikzmark{swapints}integers and one to swap \tikzmark{swapstrings}strings.
\begin{columns}
\column[t]{0.5\textwidth}
\begin{lstlisting}[frame=lrbt]
 $\tikzmark{swapint1}$void swap (int& var1, 
            int& var2) {$\tikzmark{swapint2}$
  int temp = var1;
  var1 = var2;
  var2 = temp;
}
\end{lstlisting}
\column[t]{0.5\textwidth}
\begin{lstlisting}[frame=lbrt]
 $\tikzmark{swapstring1}$void swap (string& var1, 
            string& var2) {$\tikzmark{swapstring2}$
  string temp = var1;
  var1 = var2;
  var2 = temp;
}
\end{lstlisting}
\end{columns}

The only operator needed to do a swap is
assignment (=).

\begin{tikzpicture}[overlay,remember picture]
  \balloon<1>{swapint1}{swapint2};
  \balloon<2>{swapstring1}{swapstring2};
  \path<1>[black, line width=2pt, ->] (swapints.south) edge [out=270, in=90] (balloon.north);
  \path<2>[black, line width=2pt, ->] (swapstrings.south) edge [out=270, in=90] (balloon.north);
\end{tikzpicture}

\end{frame}

%%%%%%%%%%%%%%%%%%%%%%%%%%%%%%%%%%%%%%%%%%%%%%%%%%%%%%%%%%%%


\begin{frame}[fragile]{Rule on overloading functions}
\begin{itemize}
\item Functions must differ in their signatures (ie., parameter lists).

\item It is not sufficient for the functions to differ only in the
  return type.  The compiler cannot determine which function to call
  if the function parameter data types are convertible.
\end{itemize}

\begin{lstlisting}
int function (int n);
char function (int n);

// ...

cout << $\tikzmark{overload1}$function(2);$\tikzmark{overload2}$ 
\end{lstlisting}


\begin{tikzpicture}[overlay,remember picture]
  \balloon<1>{overload1}{overload2};
  \node (comment) [comment] at (.8\textwidth, .2\textheight) {%
    Can not determine the function
  };
  \path<1>[black, line width=2pt, ->] (comment.west) edge [out=180, in=0] (balloon.east);
\end{tikzpicture}

\end{frame}

%%%%%%%%%%%%%%%%%%%%%%%%%%%%%%%%%%%%%%%%%%%%%%%%%%%%%%%%%%%%


\begin{frame}[fragile]{Template functions}

\begin{itemize}
\item C++ introduced the template as a language construct to facilitate
creation of generic functions and classes.

\item A template is instantiated with a particular data type.  (Think
  find \& replace)

\item The fully defined function is created and
  inserted into the code.
\end{itemize}

\begin{lstlisting}
// File: tswap.h

template <typename T>
void tswap (T& var1, T& var2) {
  T temp = var1;
  var1 = var2;
  var2 = temp;
}
\end{lstlisting}

\end{frame}

%%%%%%%%%%%%%%%%%%%%%%%%%%%%%%%%%%%%%%%%%%%%%%%%%%%%%%%%%%%%

\begin{frame}[fragile]{Template functions}

\begin{lstlisting}
// File: $\tikzmark{template1}$tswap.h$\tikzmark{template2}$
$\tikzmark{template3}$template <$\tikzmark{template4}$typename$\tikzmark{template5}$ $\tikzmark{template6}$T$\tikzmark{template7}$>$\tikzmark{template8}$
void tswap (T& var1, T& var2) {
  T temp = var1;
  var1 = var2;
  var2 = temp;
}
\end{lstlisting}
\begin{tikzpicture}[overlay,remember picture]
  \highlight<1>{template3}{template8};
  \highlight<2>{template4}{template5}; 
  \highlight<3>{template6}{template7}; 
  \highlight<4>{template1}{template2}; 
\end{tikzpicture}

\vspace{-2em}

\begin{itemize}
\item The syntax \lstinline$template<typename T>$ precedes the function
  definition. T is a sample identifier - use any identifier you like.

\pause

\item Alternative syntax is \lstinline$template <class T>$. This use of the word
  class has nothing to do with the C++ class construct.

\pause

\item Type T is used within the function code.  It will be
  instantiated with a real data type.

\pause

\item Templated functions and classes belong in header files because they
  are not fully defined until they are instantiated.
\end{itemize}

\end{frame}

%%%%%%%%%%%%%%%%%%%%%%%%%%%%%%%%%%%%%%%%%%%%%%%%%%%%%%%%%%%%

\begin{frame}[fragile]{Using tswap function}
\begin{lstlisting}[style=eclipse]
#include <iostream>
#include <string>
#include "tswap.h"
using namespace std;

int main () {
  int n1 = 4, n2 = 7;
  string w1 = "U", w2 = "D";
  cout << n1 << n2 << endl;
  $\tikzmark{tswap1}$tswap (n1, n2);$\tikzmark{tswap2}$
  cout << n1 << n2 << endl;
  cout << w1 << w2 << endl;
  $\tikzmark{tswap3}$tswap (w1,w2);$\tikzmark{tswap4}$
  cout << w1 << w2 << endl;
  return 0;
}
\end{lstlisting}

\begin{tikzpicture}[overlay,remember picture]
  \highlight<1,2>{tswap1}{tswap2}
  \highlight<1,2>{tswap3}{tswap4}
  \node (comment) [comment, text width=4cm] at (.75\textwidth, .7\textheight) {%
    T is replaced by \only<1>{\lstinline$int$}\only<2>{\lstinline$string$}
  };
  \path<1>[black, line width=2pt, ->] (comment.south) edge [out=270, in=0] (tswap2.east);
  \path<2>[black, line width=2pt, ->] (comment.south) edge [out=270, in=0] (tswap4.east);
\end{tikzpicture}

\end{frame}

%%%%%%%%%%%%%%%%%%%%%%%%%%%%%%%%%%%%%%%%%%%%%%%%%%%%%%%%%%%%


\begin{frame}[fragile]{Minimum template function}

An example of a template function that finds the smallest element in
an array.

\begin{lstlisting}
// PRE: =, < is defined on T
// PRE: size > 0
// POST: return smallest element in array
template <typename T>
T tmin (T array[], int size) {
  T min = array[0];

  for (int k = 0; k < size; k++)
    if (array[k] < min)
      min = array[k];

  return min;
}
\end{lstlisting}

\end{frame}

%%%%%%%%%%%%%%%%%%%%%%%%%%%%%%%%%%%%%%%%%%%%%%%%%%%%%%%%%%%%

\begin{frame}[fragile]{Memory size template function}

An example of a template function that determines if two objects are
the same size.

\begin{lstlisting}[style=eclipse]
// POST: return true if sizes are same
template <typename T1, typename T2>
bool tEqual (const T1& var1, const T2& var2) {
  return (sizeof(var1) == sizeof(var2));
}
\end{lstlisting}
\end{frame}

%%%%%%%%%%%%%%%%%%%%%%%%%%%%%%%%%%%%%%%%%%%%%%%%%%%%%%%%%%%%

\begin{frame}[fragile]{Templated class}

Classes can also be templated in C++.

\begin{lstlisting}[style=eclipse]
template <typeneame T>
class TStack {
public:
  TStack();
  TStack (int s);
  $\tikzmark{cc1}$TStack(const TStack& other);$\tikzmark{cc2}$
  ~TStack();
  void push (const T& item);
  T top () const;
  /* Additional methods */
private:
  T * array;
  int capacity;
  int size;
};
\end{lstlisting}


\begin{tikzpicture}[overlay,remember picture]
  \balloon<1>{cc1}{cc2};
  \node (comment) [comment, text width=5cm] at (.75\textwidth, .3\textheight) {%
    The copy constructor uses pass-by-reference instead of pass-by-value. Why?
  };
  \path<1>[black, line width=2pt, ->] (comment.north) edge [out=90, in=0] (balloon.east);
\end{tikzpicture}

\end{frame}

%%%%%%%%%%%%%%%%%%%%%%%%%%%%%%%%%%%%%%%%%%%%%%%%%%%%%%%%%%%%


\begin{frame}[fragile]{Constructor examples}

Each constructor is defined as a template.

\begin{lstlisting}
template <typename T>
TStack<T>::TStack () {
  capacity = 10;
  array = $\bh$new T[capacity];$\eh\tikzmark{new1}$
  size = 0;
}

template <typename T>
TStack<T>::TStack (int s) { 
  capacity = s;
  array = $\bh$new T[capacity];$\eh\tikzmark{new2}$
  size = 0;
}
\end{lstlisting}


\begin{tikzpicture}[overlay,remember picture]
  \node (comment) [comment, text width=5cm] at (.8\textwidth,
  .35\textheight) { Note: The capacity instance variable is OK even
    though it is not a constant because memory is allocated on the
    heap.
  };
  \path<1>[black, line width=2pt, ->] (comment.west) edge [out=180, in=0] (new1.east);
  \path<1>[black, line width=2pt, ->] (comment.south) edge [out=270, in=0] (new2.east);
\end{tikzpicture}

\end{frame}

%%%%%%%%%%%%%%%%%%%%%%%%%%%%%%%%%%%%%%%%%%%%%%%%%%%%%%%%%%%%


\begin{frame}[fragile]{Copy Constructor}

\begin{itemize}
\item The copy constructor uses pass-by-reference. 
\item The constructor has access to all of the private variables in the other instance.
\end{itemize}

\begin{lstlisting}[style=eclipse]
template <typename T>
TStack<T>::TStack(const TStack& other) {
  capacity = other.capacity;
  size = other.size;
  array = new T[capacity];
  for (int k = 0; k < size; k++)
  array[k] = other.array[k];
}
\end{lstlisting}

\end{frame}

%%%%%%%%%%%%%%%%%%%%%%%%%%%%%%%%%%%%%%%%%%%%%%%%%%%%%%%%%%%%


\begin{frame}[fragile]{Push method}

\begin{lstlisting}[style=eclipse]
template <typename T>
TStack<T>::push (int item) {
  $\tikzmark{push1}$if (capacity == size) {$\tikzmark{push2}$
    capacity = 2 * capacity;
    $\tikzmark{push3}$T * temp = new T[capacity];$\tikzmark{push4}$
    for (int k = 0; k < size; k++)
      $\tikzmark{push5}$temp[k] = array[k];$\tikzmark{push6}$
    $\tikzmark{push7}$delete [] array;$\tikzmark{push8}$
    array = temp;
  }
  array[size] = item;
  size++;
}
\end{lstlisting}

\begin{tikzpicture}[overlay,remember picture]
  \highlight<1>{push1}{push2};
  \highlight<2>{push3}{push4}; 
  \highlight<3>{push5}{push6}; 
  \highlight<4>{push7}{push8}; 
\end{tikzpicture}
The push method first checks if the array is full. \pause If so, it
creates a new array of twice the size, \pause copies elements, \pause and deletes
the old array.



\end{frame}

%%%%%%%%%%%%%%%%%%%%%%%%%%%%%%%%%%%%%%%%%%%%%%%%%%%%%%%%%%%%



\end{document}

%  LocalWords:  swapints swapstrings swapint swapstring ie cout tswap
%  LocalWords:  typename Templated iostream endl PRE tmin bool tEqual
%  LocalWords:  const sizeof templated typeneame TStack
